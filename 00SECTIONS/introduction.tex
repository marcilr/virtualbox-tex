%% -*- Mode: LaTeX -*-
%%
%% introduction.tex
%% Created Thu Jun 13 14:18:18 AKDT 2019
%% Copyright (C) 2019 by Raymond E. Marcil <marcilr@gmail.com>
%%
%% introduction
%%

%% ====================== Introduction ===========================
%% ====================== Introduction ===========================
\newpage
\setcounter{secnumdepth}{0}
\section{Introduction}

'VirtualBox is a powerful x86 and AMD64/Intel64
\href{https://www.virtualbox.org/wiki/Virtualization}{virtualization}
product for enterprise as well as home use.  Not only is VirtualBox
an extremely feature rich, high performance product for enterprise 
customers, it is also the only professional solution that is freely
available as Open Source Software under the terms of the GNU General
Public License (GPL) version 2. See 
"\href{https://www.virtualbox.org/wiki/VirtualBox}{About VirtualBox}"
for an introduction.\\
\\
Presently, VirtualBox runs on Windows, Linux, Macintosh, and Solaris
hosts and supports a large number of
\href{https://www.virtualbox.org/wiki/Guest_OSes}{guest operating systems}
including but not limited to Windows (NT 4.0, 2000, XP, Server 2003, Vista,
Windows 7, Windows 8, Windows 10), DOS/Windows 3.x, Linux (2.4, 2.6,
3.x and 4.x), Solaris and OpenSolaris, OS/2, and OpenBSD.\\
\\
VirtualBox is being actively developed with frequent releases and has
an ever growing list of features, supported guest operating systems
and platforms it runs on.  VirtualBox is a community effort backed by
a dedicated company: everyone is encouraged to contribute while Oracle
ensures the product always meets professional quality criteria.'\footnote{\href{https://www.virtualbox.org/}{VirtualBox: https://www.virtualbox.org/}}

